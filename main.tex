
\documentclass[10pt,a4paper]{scrartcl}

\usepackage[english]{babel}

\input{../Headerfiles/Packages}
\input{../Headerfiles/Titles}
\input{../Headerfiles/Commands}
\graphicspath{{Pictures/}}
\parindent 0pt

\title{System Identification}
\author{GianAndrea Müller}

\newtheorem{define}{Definition}

%additional commands
\newcommand{\ejon}{(e^{j\omega_n})}
\newcommand{\ejo}{(e^{j\omega})}
\newcommand{\ejzw}{(e^{j(\zeta-\omega_n)})}
\newcommand{\ejz}{(e^{j\zeta})}

\begin{document}
\begin{multicols*}{4}
\maketitle
\tableofcontents
\end{multicols*}

\begin{multicols*}{2}
\section{System Identification}
\section{Definitions}

\begin{define}
A system is said to be \textbf{time invariant} if the response to a certain input is not depending on absolute time.
\end{define}

\begin{define}
A system is said to be \textbf{linear} if its output response to a linear combination of inputs is the same as the linear combination of the output responses of the individual inputs.
\end{define}

\begin{define}
A system is said to be \textbf{causal} if the output at a certain time depends on the input up to that time only.
\end{define}

\begin{define}
A process is said to be \textbf{stationary} if it does not depend on time.
\end{define}

\section{Frequency Domain Methods}

\subsection{Sampling Operation}

\importname{Sampling with period $T$}{$y(k) = \left.y(t)\right|_{t=kT,k=0,1,2,\ldots}$}

\subsection{Fourier Series of Periodic Signals}

\important{$X(e^{j\omega_m})=\sum\limits_{k=0}^{M-1}x(k)e^{-j\omega_m k}$}

\mportant{$\omega_m=\frac{2\pi m}{M}=\omega_0$}

Non-negative frequencies are $m=0$ to $m=M/s$.

They correspond to: $\omega_m = 0,\frac{2\pi}{M},\frac{4\pi}{M},\ldots,\frac{2\pi(M/2-1)}{M},\pi$.

\begin{TDefinitionTable*}
$M$&number of samples&\\
$\omega_0=\frac{2\pi}{M}$&fundamental frequency ($y(k)$)&$\siu{\radian}$\\
$T$&sampling time&$\siu{\second}$\\
$\tau_p=MT$&period&$\siu{\second}$\\
$\omega_0=\frac{2\pi}{\tau_p}$&fundamental frequency ($y(t)$)&$\siu{\radian\per\second}$\\
\end{TDefinitionTable*}

\mportant{$0,\underbrace{\frac{2\pi}{\tau_p}}_{\substack{\text{Fundamental}\\\text{frequency}}},\underbrace{2\left(\frac{2\pi}{\tau_p}\right),\ldots,\frac{M}{2}\left(\frac{2\pi}{\tau_p}\right)}_{\text{Harmonics}}$}

\begin{define}
The highest frequency $\omega_u=\omega_{M/2}=\frac{\pi}{T}$ is called the \textbf{Nyquist frequency}.
\end{define}

\myspic{0.5}{NyquistFrequency}

\begin{TPMatlab}
%Non negative frequency vector
omega = omega_n(N,'p0');
%Discrete Fourier Transform
fft(u); %Matlab
DFT(u); %Lecture
\end{TPMatlab}

\begin{tiny}
Note that the definition of the fft in the Matlab documentation does not describe the actual implementation perfectly. The index actually runs from 0 to $N-1$ as in the definition made in class.
\end{tiny}

\section{Spectral Estimation}

\myspic{0.5}{IODiagram}

\mportname{Transfer function}{$Y(j\omega)=G(j\omega)U(j\omega)$}

\mportname{Discrete time TF}{$Y(e^{j\omega}=G(e^{j\omega}U\ejo$}

\mportant{$\frac{\omega_u}{2\pi}=\frac{r}{NT}$}

\begin{TDefinitionTable*}
$\omega_u$&input frequency&$\siu{\rad\per\second}$\\
$N$&calculation length&$\siu{\ }$\\
$T$&experiment duration?&$\siu{\second}$\\
$r$&some integer&$\siu{\ }$\\
\end{TDefinitionTable*}

\mportname{Input}{$u(k)=\alpha\cos(\omega_u k),\ k=0,1,\ldots,K-1\ \text{ with } K\geq N$}

\mportname{Output}{$y(k)=\alpha\left|G(e^{j\omega_u})\right|\cos(\omega_u k+\theta(\omega_u))+v(k)+\text{transient}$}

where $\theta(\omega_u)=arg(G(e^{j\omega_u})$

\subsection{Sinusoidal correlation methods}

Correlation functions:

\mportant{$I_c(N)=\frac{1}{N}\sum\limits_{k=0}^{N-1}y(k)\cos(\omega_uk)$}

\mportant{$I_s(N)=\frac{1}{N}\sum\limits_{k=0}^{N-1}y(k)\sin(\omega_u k)$}

To calculate those from the data:

\small
\begin{align*}
I_c(N)=\frac{\alpha}{2}\left|G(e^{j\omega_u})\right|\cos(\theta(\omega_u)+\frac{\alpha}{2}\left|G(e^{j\omega_u)}\right|\frac{1}{N}\sum\limits_{k=0}^{N-1}\cos(2\omega_u k+\theta(\omega_u))+\frac{1}{N}\sum\limits_{k=0}^{N-1}v(k)\cos(\omega_u k)
\end{align*}
\normalsize

If the noise, $v(k)$ is sufficiently uncorrelated then the variance satisfies,

\mportant{$\lim\limits_{N\rightarrow\infty}\text{var}\left\{\frac{1}{N}\sum\limits_{k=0}^{N-1}v(k)\cos(\omega_uk)\right\} =0$}

with a convergence rate of $1/N$.

Thus in the limit $N\rightarrow\infty$,

\begin{align*}
E\left\{I_c(N)\right\}&\rightarrow\frac{\alpha}{2}\left|G(e^{j\omega_u})\right|\cos(\theta(\omega_u))\\
E\left\{I_s(N)\right\}&\rightarrow-\frac{\alpha}{2}\left|G(e^{j\omega_u})\right|\sin(\theta(\omega_u))
\end{align*}

and since $\lim\limits_{N\rightarrow\infty}\ \text{var}\left\{I_c(N)\right\}=0,\quad\lim\limits_{N\rightarrow\infty}\ \text{var}\left\{I_s(N)\right\}=0$

The transfer function can be estimated via:

\important{$\hat{G}_N(e^{j\omega_u})=\frac{I_c(N)-jI_s(N)}{\alpha/2}$}

\begin{itemize}
\item Advantages
\begin{itemize}
\item Energy is concentrated at the frequencies of interest.
\item Amplitude of $u(k)$ can easily be tuned as a function of frequency.
\item Easy to avoid saturation and tune signal/noise (S/N) ratio.
\end{itemize}
\item Disadvantages
\begin{itemize}
\item A large amount of data is required.
\item Significant amount of time required for experiments.
\item Some processes won't allow sinusoidal inputs.
\end{itemize}
\end{itemize}

\section{Frequency Domain Methods}

%\textbf{Signals considered}
%
%\begin{itemize}
%\item Finite energy
%\item Periodic
%\item Random
%\item Finite length
%\end{itemize}
%
%\textbf{Signal properties of interest}
%
%\begin{itemize}
%\item Autocorrelation
%\item Crosscorrelation
%\item Frequency domain representation
%\item Spectral density (energy or power)
%\end{itemize}

\mportname{Discrete-time domain signal}{$x(k),\ k=-\infty,\ldots,\infty$}

\importname{Fourier Transform}{$X\ejo=\sum\limits_{k=-\infty}^{\infty}x(k)e^{-j\omega k}$}

\begin{itemize}
\item $X\ejo$ is $2\pi$ periodic.
\item If $\sum\limits_{k=-\infty}^\infty|x(k)|<\infty$ then $X\ejo$ converges.
\end{itemize}

\importname{Inverse Fourier Transform}{$x(k)=\frac{1}{2\pi}\int_{-\pi}^{\pi}X\ejo e^{j\omega k}d\omega$}

where $k=-\infty,\ldots,\infty$

\subsection{Finite Energy Signal}

\subsubsection{Energy Spectral Density (Finite Energy Signal)}

If $x(k)$ is a finite energy signal,

\mportant{$||x(k)||_2^2=\sum\limits_{k=-\infty}^{\infty}|x(k)|^2<\infty$}

\importname{Energy Spectral Density}{$S_x\ejo=|X\ejo|^2$}

\textbf{For all following calculations of the energy spectral density finiteness is assumed.}

\subsubsection{Autocorrelation (Finite Energy Signal)}

\important{$R_x(\tau)=\sum\limits_{k=-\infty}^{\infty}x(k)x(k-\tau),\quad \tau=-\infty,\ldots,0,\ldots,\infty$}

The spectral density is the Fourier Transform of the autocorrelation:

\mportant{$\sum\limits_{\tau=-\infty}^{\infty}R_x(\tau)e^{-j\omega\tau}=S_x\ejo$}

\begin{TPMatlab}
% autocorrelation for a non-periodic, finite energy signal
Correlation('finen',u); %Lecture
xcorr(u); %Matlab

% crosscorrelation for non-periodic, finite energy signals
Correlation('finen',u,y); %Lecture
xcorr(y,u); %Matlab
\end{TPMatlab}

\subsection{Discrete Periodic Signal}

\mportname{Periodic signal}{$x(k)=x(k+M),\quad \forall\ k\in\{-\infty,\infty\}$}

\mportname{Fundamental frequency}{$\omega_0=\frac{2\pi}{M}$}

\begin{itemize}
\item There are only $M$ unique harmonics of the sinusoid $e^{j\omega_0}$.
\item The non-negative harmonic frequencies are,

\mportant{$e^{jn\omega_0},\ n=0,1,\ldots,M/2$}
\end{itemize}

\begin{TPMatlab}
u_period = sqrt(variance)*randn(N,1)+bias;
u = repmat(u_period,periods,1);
\end{TPMatlab}

\subsubsection{Discrete Fourier Series (Discrete Periodic Signal)}

\important{$X(e^{j\omega_n})=\sum\limits_{k=0}^{N-1}x(k)e^{-j\omega_n k},\ \text{where}\ \omega_n =\frac{2\pi n}{N}=n\omega_0$}

\importname{Inverse Transform}{$x(k)=\frac{1}{N}\sum\limits_{k=0}^{N-1}X(e^{j\omega_n})e^{j\omega_nk}$}

\subsubsection{Autocorrelation (Discrete Periodic Signal)}

\important{$R_x(\tau)=\frac{1}{N}\sum\limits_{k=0}^{N-1}x(k)x(k-\tau)$}

The Fourier transform of $R_x(\tau)$ is now defined as the \textbf{power spectral density}, since it is normalized with the signal length.

\important{$\phi_x(e^{j\omega_n})=\sum\limits_{\tau=0}^{N-1}R_x(\tau)e^{-j\omega_n\tau}=\frac{1}{N}|X(e^{j\omega_n})|^2$}

The energy in a single period is:

\mportant{$\sum\limits_{k=0}^{N-1}|x(k)|^2=\sum\limits_{n=0}^{N-1}\phi_x(e^{j\omega_n})$}

\begin{TPMatlab}
Correlation('periodic',u);
\end{TPMatlab}

\subsubsection{Cross-Correlation (Discrete Periodic Signal)}

\important{$R_{yu}(\tau)=\frac{1}{N}\sum\limits_{k=0}^{N-1}y(k)u(k-\tau)$}

The Fourier transform of $R_{yu}(\tau)$ is now defined as the \textbf{cross-spectral density}.

\important{$\phi_{yu}(e^{j\omega_n})=\sum\limits_{\tau = 0}^{N-1}R_{yu}(\tau)e^{-j\omega_n\tau}) = \frac{1}{N}Y(e^{j\omega_n})U^\ast(e^{j\omega_n})$}

\begin{TPMatlab}
Correlation('periodic',u,y);
\end{TPMatlab}

\subsection{Random Signal}

Normally distributed noise:

\important{$e(k)\in\mathcal{N}(0,\lambda)\Rightarrow\begin{cases}\expe{e(k)}=0\text{ (zero mean)}\\\expe{|e(k)|^2}=\lambda\text{ (variance)}\end{cases}$}

The $e(k)$ are independent and identically distributed (i.i.d.).

\begin{TPMatlab}
standard_deviation = 2;
variance = standard_deviation^2;
bias = 0;
N = 1024;
u = randn(N,1)*standard_deviation + bias;
\end{TPMatlab}

\subsubsection{Autocovariance (Random Signal)}

\begin{align*}
R_x(\tau) &= \expe{x(k)x(k-\tau)}\\
&=\expe{x(k)x^\ast(k-\tau)} \text{ (in the complex case)}\\
&=\expe{x(k)x^\ast(x-\tau)} \text{ (in the multivariable case)}
\end{align*}

General (non-stationary, non-zero mean) case:

\begin{align*}
R_x(s,t)&=\expe{(x(s)-\expe{x})(x(t)-\expe{E})}\\
&= \expe{x(s)x(t)}\text{ (if zero mean)}\\
&= R_x(s-t)\text{ (if stationary)}
\end{align*}

Further properties are

\begin{itemize}
\item $R_x(-\tau)=R_x^\ast(\tau)$
\item $R_x(0)\geq |R_x(\tau)|\ \forall\tau>0$
\end{itemize}

\begin{TPMatlab}
xcorr(u)/N; %Lecture 3.37
\end{TPMatlab}

\subsubsection{Power Spectral Density (Random Signal)}

\important{$\phi_x\ejo:=\sum\limits_{\tau=-\infty}^{\infty}R_x(\tau)e^{-j\omega\tau}$ where $\omega\in[-\pi,\pi)$}

For a zero-mean random signal:

\mportant{$\lim\limits_{N\rightarrow\infty}\frac{1}{N}\sum\limits_{k=0}^{N-1}|x(k)|^2=\var{x(k)}=\frac{1}{2\pi}\int_{-\pi}^{\pi}\phi_x\ejo d\omega$}


Further properties are

\begin{itemize}
\item $\phi_x\ejo\in\mathbb{R}$
\item $\phi_x\ejo\geq 0\ \forall\ \omega$
\item $\phi_x\ejo=\phi_x(e^{-j\omega})$ for all real-valued $x(k)$
\end{itemize}

\begin{TPMatlab}
fft(xcorr(u)/N)) %Lecture 3.37 (?)
\end{TPMatlab}

\subsubsection{Cross-Covariance (Random Signal)}

\important{$R_{yu}(\tau)=\expe{(y(k)-\expe{y(k)})(u(k-\tau)-\expe{u(k)}}$}

For zero mean signals:

\mportant{$R_{yu}(\tau)=\expe{y(k)u(k-\tau)}$}

Joint stationarity is required to make the definition dependent on $\tau$ only.

If $R_{yu}(\tau)=0$ for all $\tau$ then $y(k)$ and $u(k)$ are uncorrelated.

\begin{TPMatlab}
xcorr(u,y)/N; %Lecture 3.37 (?)
\end{TPMatlab}

\subsubsection{Cross Power Spectral Density (Random Signal)}

\important{$\phi_{yu}\ejo=\sum\limits_{\tau=-\infty}^\infty R_{yu}(\tau)e^{-j\omega\tau},\ \omega\in[-\pi,\pi)$}

The inverse is,

\mportant{$R_{yu}(\tau)=\frac{1}{2\pi}\int_{-\pi}^\pi\phi_{yu}\ejo e^{j\omega\tau}d\omega$}

\begin{TPMatlab}
fft(xcorr(y,u)/N); %Lecture 3.37 (?)
\end{TPMatlab}

\subsection{Finite Length Signal}

\subsubsection{Discrete-Fourier Transform (Finite Length Signal)}

\important{$X_N(e^{j\omega_n})=\sum\limits_{k=0}^{N-1}x(k)e^{-j\omega_nk},$ where $\omega_n=\frac{2\pi n}{N}$}

The inverse DFT is

\mportant{$x(k)=\frac{1}{N}\sum\limits_{n=0}^{N-1}X_N(e^{j\omega_n})e^{j\omega_n k},\quad k=0,\ldots, N-1$}

\subsubsection{Periodogram (Finite Length Signal)}

\important{$\frac{1}{N}\left|V_N\ejo\right|^2$}

An asymptotically unbiased estimator of the spectrum is

\mportant{$\lim\limits_{N\rightarrow\infty}\expe{\frac{1}{N}|V_N(e^{j\omega}|^2}=\phi_v(\omega)$}

This assumes that the autocorrelation decays quickly enough:

\mportant{$\lim\limits_{N\rightarrow\infty}\frac{1}{N}\sum\limits_{\tau=-N}^N|\tau R_v(\tau)| =0$}

\section{ETFE}

\myspic{0.5}{IODiagram}

Linear, time-invariant system, $g(l)$:

\mportant{$y(k)=\sum\limits_{l=0}^\infty g(l)u(k-l)+v(k),\quad k=0,1,\ldots$}

Assumptions:

\begin{enumerate}
\item causal system: $g(l) = 0,\forall l<0$
\item noise: $E\{v(k)\} = 0$, zero mean, stationary
\end{enumerate}

Given $\{u(k),y(k)\}$ find an estimate $\hat{G}\ejo$ such that it fits the $G\ejo$.

\importname{Bias}{Bias($\hat{G}))G-E\{\hat{G}\}$}

\importname{Variance}{var($(\hat{G})=E\left\{|\hat{G}-E\{\hat{G}\}|^2\right\}$}

\importname{Mean-square error}{MSE$(\hat{G})=E\left\{|G-\hat{G}|^2\right\}$}

Note that MSE$(\hat{G})=$var$(\hat{G})+$Bias$^2(\hat{G})$.

\subsection{Input-output relationship}

For finite energy signals:

\mportant{$y(k) = \sum\limits_{l=0}^\infty g(l)u(k-l)+v(k)$}

\mportant{$Y\ejo=G\ejo U\ejo+V\ejo$}

which in the idealized case leads to:

\important{$\frac{Y\ejo}{U\ejo}=G\ejo+\frac{V(e^j\omega)}{U\ejo}\approx G\ejo$}

In reality we only have $N$ samples:

\mportant{$\underbrace{Y_N(e^{j\omega_n})}_{\text{length-N DFT}} = \sum\limits_{k=0}^{N-1}y(k) e^{-j\omega_n k}\approx \sum\limits_{k=-\infty}^\infty y(k)e^{-j\omega_n k}=Y(e^{j\omega_n}$}

\mportant{$\underbrace{U_N(e^{j\omega_n})}_{\text{length-N DFT}}=\sum\limits_{k=0}^{N-1}u(k)e^{-j\omega_n k}\approx \sum\limits_{k=-\infty}^\infty u(k)e^{-j\omega_n k}=U(e^{j\omega_n})$}

\importname{ETFE}{$\hat{G}_N(e^{j\omega_n}):=\frac{Y_N(e^{j\omega_n})}{U_N(e^{j\omega_n})}$}

\subsection{Periodic input case}

Period $M$ inputs: $u(k)=u(k+M)$

If $sM = N$ for an integer $s$, the fourier series over N samples is equal to the real fourier series!

\mportant{$U_N(e^{j\omega_n}) = U(e^{j\omega_n})\forall \omega_n=\frac{2\pi n}{N},\ n=0,\ldots,N-1$}

Then

\mportant{$Y_N(e^{j\omega_n})=G(e^{j\omega_n})U_N(e^{j\omega_n})+V_N(e^{j\omega_n})$}

\mportant{$\hat{G}_N(e^{j\omega_n})=G(e^{j\omega_n})+\frac{V_N(e^{j\omega_n})}{U_N(e^{j\omega_n})}$}

%\subsubsection{Error properties}

\textbf{Bias:}

\mportant{$E\{\hat{G}_N(e^{j\omega_n})\}=G(e^{j\omega_n})+E\left\{\frac{V_N(e^{j\omega_n})}{U_N(e^{j\omega_n})}\right\}=G(e^{j\omega_n})$}

when assuming zero mean noise. Thus for periodic inputs with $N$ being an integer number of periods, \textbf{the ETFE is unbiased}.

\textbf{Variance:}

For the unbiased case:

\renewcommand{\expe}[1]{\text{E}\left\{#1\right\}}

\mportant{$E\left\{|\hat{G}_N(e^{j\omega_n})-G(e^{j\omega_n})|^2\right\}=\frac{\phi_v(e^{j\omega_n})+\frac{2}{N}c}{\frac{1}{N}|U_N(e^{j\omega_n})|^2}$}

where $|c|\leq C=\sum\limits_{\tau =1}^\infty |\tau R_v(\tau)|$ is assumed to be finite.

For estimates at different frequencies ($\omega_n\neq\omega_i$):

\mportant{$E\left\{(\hat{G}_N(e^{j\omega_n})-G(e^{j\omega_n}))(\hat{G}_N(e^{-j\omega_i)}-G(e^{-j\omega_i}))\right\}=0$}

\textbf{Transient responses:}

Initial transient corrupts the measurement

\mportant{$y(k)=G(u_{periodic}(k)W_{[0,N-1]}(k))+v(k)$}

with the window function:

\mportant{$W_{[0,N-1]}(k)=\begin{cases}
1&\text{ if } 0\leq0<N\\
0&\text{ otherwise}\end{cases}$}

For all outputs up to time $k=N-1$

\mportant{$y(k)=Gu_{periodic}(k)-\underbrace{G(u_{periodic}W_{(-\infty,-1)})}_{r(k)}+v(k)$}

\mportant{$Y_N(e^{j\omega_n})=G(e^{j\omega_n})U_N(e^{j\omega_n})+R_N(e^{j\omega_n})+V_N(e^{j\omega_n})$}

The input in negative time, which is present in a ideal periodic input, and missing in a real periodic input, has an influence on positive time, which is described by $r(k)$.

\vspace{3ex}

When using a periodic signal multiple times the resulting DFT does not contain more information, since in a periodic signal there are only a certain number of frequencies contained, but the energy in those frequencies increases!

\textbf{Transient bias error:}

\mportant{$\hat{G}\ejon=\frac{Y_N\ejon}{U_N\ejon}=G\ejon+\frac{R_N\ejon}{U_N\ejon}+\frac{V_N\ejon}{U_N\ejon}$}

\textbf{For periodic $u(k)$}

As $N=mM,m\rightarrow\infty$

\mportant{$|U_N\ejon|=m|U_M\ejon|$}

\textbf{For random $u(k)$}

As $N\rightarrow\infty$

\mportant{$E\{|U_N\ejon|\}\rightarrow\sqrt{N}\sqrt{\phi_u\ejon}$}

Thus

\important{$\left|\frac{R_N\ejon}{U_N\ejon}\right|\rightarrow 0$ with rate $\begin{cases}
\frac{1}{N}&\text{ for periodic input}\\
\frac{1}{\sqrt{N}}&\text{ for random inputs}\end{cases}$}

A fix for getting rid of the influence of the transient response: Get rid of the first period.

\begin{TPMatlab}
u_av = Averaging(u,N,n_periods,loose_transient);
\end{TPMatlab}

\subsection{Spectral Transformations}

If $v(k) = 0$

\mportant{$\phi_y\ejon=G\ejon\phi_u\ejon G^T\ejon$}

where $G^T\ejon$ is the complex conjugate of $G\ejon$.

If $v(k)\neq 0$ and uncorrelated

\mportant{$\phi_y\ejon = |G\ejon|^2\phi_u\ejon+|H\ejon|^2$}

But this approach has no more phase information. For that reason use the cross spectrum:

\mportant{$\phi_{yu}\ejon=G\ejon\phi_u\ejon+\phi_{uv}\ejon = G\ejon\phi_u\ejon$}

if $u(k)$ and $v(k)$ are uncorrelated.

\importname{Spectral estimation methods}{$\hat{G}\ejo = \frac{\hat{\phi}_{yu}\ejon}{\hat{\phi}_u\ejon}$}

where

\mportant{$\phi_y\ejon=|G\ejon|^2\phi_u\ejon+\phi_v\ejon$}

\mportant{$\phi_{yu}\ejon=G\ejon\phi_u\ejon$}

\subsection{Approaches to spectral estimation}

\subsubsection{Spectral estimation (Periodogram)}

The periodogram is an asymptotically unbiased estimator of the spectrum given $\lim\limits_{n\rightarrow\infty}\frac{1}{N}\sum\limits_{\tau=-\N}^N|\tau R_v(\tau)|=0$

\mportname{Periodogram}{$\frac{1}{N}|V_N\ejon|^2$}

\important{$\lim\limits_{N\rightarrow\infty}E\left\{\frac{1}{N}|V_N\ejon|^2\right\}=\phi_v\ejon$}

which is under the assumption

\mportant{$\lim\limits_{N\rightarrow\infty}\frac{1}{N}\sum\limits_{\tau=-N}^{N}|\tau R_v(\tau)|=0$}

\subsubsection{Spectral estimation (via covariances)}

The autocovariance of the noise for stochastic $v(k)$ is described as:

\mportant{$\hat{R}_v(\tau)=\begin{cases}
\frac{1}{N-|\tau|}\sum\limits_{k=\tau}^{N_1}v(k)v(k-\tau),&\text{for }\tau\geq 0\\
\frac{1}{N-|\tau|}\sum\limits_{k=0}^{N+\tau-1}v(k)v(k-\tau),&\text{for }\tau<0
\end{cases}$}

This is an unbiased estimator of $R_v(\tau)$: $E\{\hat{R}_v(\tau)=R_v(\tau)\}$

\important{$\hat{\phi}_v\ejon=\sum\limits_{\tau=-N+1}^{N-1}\hat{R}_v(\tau)e^{-j\omega \tau}$}

Thus the spectral estimate is:

\mportant{$\hat{\phi}_v\ejo = \sum\limits_{\tau = -N+1}^{N-1}\hat{R}_v(\tau)e^{-j\omega\tau}$}

\begin{TPMatlab}
xcorr(a-mean(a))-xcov(a) == 0 %true
\end{TPMatlab}

\subsubsection{Spectral estimation (periodic signals)}

Periodic signal $x(k)$ with period $M$, $N=mM$ for some integer $m$

\mportant{$R_x(\tau)=\frac{1}{M}\sum\limits_{k=0}^{M-1}x(k)x(k-\tau)$}

The power spectral density can be calculated and is equal to the periodogram:

\important{$\phi_x\ejon=\sum\limits_{\tau = 0}^{M-1}R_x(\tau)e^{-j\omega_n\tau}=\frac{1}{M}\left|X_M\ejon\right|^2$}

\begin{TPMatlab}
Ru = Correlation('periodic',u);
\end{TPMatlab}

\subsubsection{Spectral estimation (more general case)}

Alternative autocorrelation estimate:

\mportant{$\hat{R}_x(\tau)=\begin{cases}
\frac{1}{N}\sum\limits_{k=\tau}^{N-1}x(k)x(k-\tau),&\text{ for }\tau\geq 0\\
\frac{1}{N}\sum\limits_{k=0}^{N+\tau-1}x(k)x(k-\tau),&\text{ for }\tau<0\end{cases}$}

Periodic $x(k)$: unbiased (exact) if $N=mM$

\vspace{3ex}

Random $x(k)$ biased $\expe{\hat{R}_x(\tau)}=\frac{N-|\tau|}{N}R_x(\tau)$.

asymptotically biased as $N\rightarrow\infty,\tau/N\rightarrow 0$

\begin{TPMatlab}
Ru = xcorr(u)/N;
\end{TPMatlab}

\section{Averaging and Smoothing}

Multiple experiments $u_r(k),y_r(k), r=1\ldots,R, k=0,\ldots, K-1$

\mportant{$\hat{G}\ejon=\sum\limits_{r=1}^R\alpha_r\hat{G}_r\ejon$}

where $\sum\limits_{r=1}^G\alpha_r = 1$ and for calculating the average $\alpha_r =\frac{1}{R}$.

\vspace{3ex}

The averaging can be optimized by selecting $\alpha_r$ such that the variance $\sigma_r^2\ejon$ is minimized.

\mportant{$\var{\hat{G}\ejon}=\var{\sum\limits_{r=1}^R\alpha_r\ejon \hat{G}_r\ejon}=\sum\limits_{r=1}^R\alpha_r^2\sigma_r^2\ejon$}

This is minimized by

\important{$\alpha_r\ejon=\frac{1/\sigma_r^2\ejon}{\sum\limits_{r=1}^T1/\sigma_r^2\ejon}$}

Thus the signal is weighted inversely proportional to the variance.

\vspace{3ex}

Thus if $\var{\hat{G}_r\ejon}=\frac{\phi_v\ejon}{\frac{1}{N}|U_r\ejon|^2}$ then $\alpha_r\ejon=\frac{|U_r\ejon|^2}{\sum\limits_{r=1}^R|U_r\ejon|^2}$.

\vspace{3ex}

The best result is obtained if the input is the same for all $r$, which will lead to a reduction of the variance as follows:

\mportant{$\var{\hat{G}\ejon}=\frac{\var{\hat{G}_r\ejon}}{R}$}

Biased estimates will reduce the improvement in variance.

\begin{itemize}
\item Since we are adding complex numbers the magnitude of the average is not equal to the average of the magnitudes $r_i$.
\end{itemize}

\subsection{Bias-variance trade-offs in data record splitting}

Divide a data record into smaller parts for averaging:

\mportant{$\{u(k),y(k)\}, k=0,\ldots,K-1$}

Choose $R$ records and calculation length $N$, such that $NR\leq K$:

\mportant{$u_r(n) = u(rN+n)$}

And average the resulting estimates:

\important{$\hat{G}\ejon=\frac{1}{R}\sum\limits_{r=0}^{R-1}\hat{G}_r\ejon=\frac{1}{R}\sum\limits_{r=0}^{R-1}\frac{\hat{Y}_r\ejon}{\hat{U}_r\ejon}$}

As $R$ increases:
\begin{itemize}
\item The number of points calculated, $N$ decreases.
\item The variance decreases (by up to $1/R$).
\item The bias increases (due to non-periodicity transients).
\end{itemize}

Mean-square error
\begin{itemize}
\item Transient bias grows linearly with the number of data splits.
\item Variance decays with a rate of up to $1/$(number of averages).
\end{itemize}

What if there is no option of running periodic input experiments? $\rightarrow$ exploit the assumed smoothness of the underlying system.

\subsection{Smoothing the ETFE}

Assume the true system to be close to constant for a range of frequencies: $G(e^{j\omega_{n+r}})\approx G(e^{j\omega_n})$ for $r=0,\pm 1,\ldots,\pm r$.

The minimum variance smoothed estimate is:

\mportant{$\tilde{G}_N\ejon=\frac{\sum\limits_{r=-R}^R\alpha_r\hat{G}_N(e^{j\omega_{n+r}})}{\sum\limits_{r=-R}^R\alpha_r},\qquad \alpha_r=\frac{\frac{1}{N}|U_N(e^{j\omega_{r+n}})|^2}{\phi_v(e^{j\omega_{n+r}})}$}

The summation above can then be approximated by an integral:

\mportant{$\approx \frac{\int_{\omega_{n-r}}^{\omega_{n+r}}\alpha\ejz\hat{G}_N\ejz d\zeta}{\int_{\omega_{n-r}}^{\omega_{n+r}}\alpha\ejz d\zeta},\qquad \text{with } \alpha\ejz=\frac{\frac{1}{N}|U_N\ejz|^2}{\phi_v\ejz}$}

Which can be reformulated using a smoothing window:

\mportant{$\tilde{G}_N\ejon=\frac{\frac{1}{2\pi}\int_{-\pi}^\pi W_\gamma\ejzw\alpha\ejz\hat{G}_N\ejz d\zeta}{\frac{1}{2\pi}\int_{-\pi}^\pi W_\gamma\ejzw\alpha\ejz d\zeta}\qquad\text{with }\alpha\ejz=\frac{\frac{1}{N}|U_N\ejz|^2}{\phi_v\ejz}$}

\subsubsection{Assumptions on $\phi_v\ejo$}

Assume $\phi_v\ejo$ is also a smooth function of frequency.

\mportant{$\frac{1}{2\pi}\int_{-\pi}^\pi W_\gamma\ejzw\left|\frac{1}{\phi_v\ejz}-\frac{1}{\phi\ejon}\right| d\zeta\approx 0$}

Then use,

\mportant{$\alpha\ejz=\frac{\frac{1}{N}|U_N\ejz|^2}{\phi_v\ejon}$}

to get

\important{$\tilde{G}_N\ejon=\frac{\frac{1}{2\pi}\int_{-pi}^\pi W_\gamma(e^{-j(\zeta-\omega_n)})\frac{1}{N}|U_N\ejz|^2\hat{G}_N\ejz d\zeta}{\frac{1}{2\pi}\int_{-\pi}^\pi W_\gamma\ejzw\frac{1}{N}|U_N\ejz^2d\zeta}$}

The wider the frequency window (decreasing $\gamma$)

\begin{itemize}
\item the more adjacent frequencies included in the smoothness estimate.
\item the smoother the result.
\item the lower the noise induced variance.
\item the higher the bias.
\end{itemize}

\subsubsection{Characteristic windows}

\importname{Bartlett}{$W_\gamma\ejo=\frac{1}{\gamma}\left(\frac{\sin\gamma\omega/2}{\sin\omega/2}\right)^2$}

\importname{Hann}{$W_\gamma\ejo=\frac{1}{2}D_\gamma(\omega)+\frac{1}{4}D_\gamma(\omega-\pi/\gamma)+\frac{1}{4}D_\gamma(\omega+\pi/\gamma)$}

where 

\mportant{$D_\gamma(\omega)=\frac{\sin\omega(\gamma+0.5)}{\sin\omega/2}$}

\textbf{Properties of window functions:}

\begin{itemize}
\item $\frac{1}{2\pi}\int_{-\pi}^\pi W_\gamma\ejz d\zeta=1$
\item $\int_{-\pi}^\pi\zeta W_\gamma\ejz d\zeta=0$
\item $M(\gamma):=\int_{-\pi}^\pi\zeta^2W_\gamma\ejz d\zeta$
\item $\bar{W}(\gamma):=2\pi\int_{-\pi}^\pi W_\gamma^2\ejz d\zeta$
\end{itemize}

\begin{tabular}{lll}
Bartlett&$M(\gamma)=\frac{2.78}{\gamma},$&$\bar{W}(\gamma)\approx 0.67\gamma\quad$(for $\gamma>5$)\\
Hamming&$M(\gamma)=\frac{\pi^2}{2\gamma^2},$&$\bar{W}(\gamma)\approx 0.75\gamma\quad$(for $\gamma>5$)
\end{tabular}

\begin{itemize}
\item $M(\gamma)$ gives an idea of the bias effect.
\item $\bar{W}(\gamma)$ gives an idea of the variance effect.
\end{itemize}

\subsubsection{Asymptotic bias properties}

\mportant{$\expe{\tilde{G}\ejon-\expe{G\ejon}}=\expe{\tilde{G}\ejon-G\ejon}=M(\gamma)\left(\frac{1}{2}\underbrace{G''\ejon}_\text{curvature}+\underbrace{G'\ejon}_{slope}\frac{\phi_u'\ejon}{\phi_u\ejon}\right)+H.O.T.$}

Increasing $\gamma$

\begin{itemize}
\item makes the frequency window smaller.
\item averages over fewer frequency values.
\item makes $M(\gamma)$ smaller
\item reduces the bias of the smoothed estimate $\tilde{G}\ejon$
\end{itemize}

\subsubsection{Asymptotic variance properties}

\mportant{$\expe{(\tilde{G}\ejon-\expe{\tilde{G}\ejon})^2}=\frac{1}{N}\bar{W}(\gamma)\frac{\phi_v\ejon}{\phi_u\ejon}+H.O.T.$}

Increasing $\gamma$

\begin{itemize}
\item makes the frequency window narrower.
\item averages over fewer frequency values.
\item makes $\bar{W}_\gamma$ larger.
\item increases the variance of the smoothed estimate $\tilde{G}\ejon$.
\end{itemize}

\subsubsection{Asymptotic MSE properties}

\mportant{$\expe{|\tilde{G}\ejon-G\ejon|^2}\approx M^2(\gamma)|F\ejon|^2+\frac{1}{N}\bar{W}(\gamma)\frac{\phi_v\ejon}{\phi_u\ejon}$}

where 

\mportant{$F\ejon=\frac{1}{2}G''\ejon+G'\ejon\frac{\phi'_u\ejon}{\phi_u\ejon}$}

If $M(\gamma)=M/\gamma^2$ and $\bar{W}(\gamma)=\bar{W}\gamma$ then MSE is minised by:

\important{$\gamma_{optimal}=\left(\frac{4M^2|F\ejon|^2\phi_u\ejon}{\bar{W}\phi_v\ejon}\right)^{1/5}N^{1/5}$}

and 

\mportant{MSE at $\gamma_{optimal}\approx CN^{-4/5}$}

\section{Windowing and Input Signals}

\mportant{$\phi_{yu}\ejo=G\ejo\phi_u\ejo$}

\important{$\hat{G}\ejon=\frac{\hat{\phi}_{yu}\ejon}{\hat{\phi}_u\ejon}$}

Recall that the smoothed ETFE is:

\mportant{$\tilde{G}_N\ejon=\frac{\frac{1}{2\pi}\int_{-\pi}^\pi W_\gamma\ejzw\frac{1}{N}|U_N\ejz|^2\hat{G}_N\ejz d\zeta}{\frac{1}{2\pi}\int_{-\pi}^\pi W_\gamma \ejzw\frac{1}{N}|U_n\ejz|^2d\zeta}$}

The denominator term approaches $\frac{1}{2\pi}\int_{-\pi}^\pi W_\gamma\ejzw \phi\ejon d\zeta$ as $N\rightarrow \infty$. 

\vspace{3ex}

If in addition $W_\gamma\ejo$ is concentrated around $\zeta = 0$ (i.e. $\gamma/N\rightarrow 0$) then the denominator term approaches $\phi_u\ejon$ as $N\rightarrow\infty$.

\vspace{3ex}

This motivates the smoothed spectral estimate:

\important{$\tilde{\phi}_u\ejon=\frac{1}{2\pi}\int_{-\pi}^\pi W_\gamma\ejzw\frac{1}{N}|U_N\ejo|^2d\zeta$}

Similarly the numerator approaches $\phi_{yu}$ as $N\rightarrow\infty$:

\important{$\tilde{\phi}_{yu}\ejon=\frac{1}{2\pi}\int_{-\pi}^\pi W_\gamma\ejzw\frac{1}{N}|U_N\ejo|^2\hat{G}_N\ejo d\zeta$}

For this reason the smoothed ETFE is equal to the smoothed spectral estimate for $N\rightarrow \infty$.

\subsection{Frequency domain smoothing in matlab}

\begin{TPMatlab}
gamma = 80;
U = fft(u); Y = fft(y);
G_est = Y./U;
G_est_smooth = G_est*0;

[om,Wg] = WfHann(g,N);
%shift to start at zero
zidx = find(om == 0);
omega = [om(zidx:N);om(1:zidx-1)];
Wg = [Wg(zidx:N) Wg(1:zidx-1)];
%variance weighting
a = U.*conj(U);
\end{TPMatlab}

\begin{TPMatlab}
for wn = 1:N
	%reset normalization
	Wnorm = 0;
    for xi = 1:N
    	%wrap window index
        widx = mod(xi-wn,N)+1;
        G_est_smooth(wn) = G_est_smooth(wn) +...
            Wg(widx)*G_est(xi)*a(xi);
        Wnorm = Wnorm + Wg(widx)*a(xi);
    end
    %weigh normalisation
    G_est_smooth(wn) = G_est_smooth(wn)/Wnorm;
end
\end{TPMatlab}

\subsection{Time domain windows}

Define, via the inverse Fourier transform a time domain window:

\mportant{$\omega_\gamma(\tau)=\frac{1}{2\pi}\int_{-\pi}^\pi W_\gamma\ejo e^{j\zeta\tau}d\zeta$}

Then the smoothed input spectral estimate $\tilde{\phi}_u\ejon$ is:

\mportant{$\frac{1}{2\pi}\int_{-\pi}^\pi W_\gamma\ejzw\frac{1}{N}|U_N\ejo|^2 d\zeta\approx\sum\limits_{\tau-\infty}^\infty \omega_\gamma(\tau)\hat{R}_u(\tau)e^{-j\tau\omega_n}$}

where

\mportant{$\omega_\gamma=\begin{cases}0 &\text{for }\tau<-\gamma\\>0&\text{for }-\gamma\leq\tau\leq\gamma\\0&\text{for }\tau>\gamma\end{cases}$}

where often $\gamma<< N$, which enables the faster calculated redefinition:

\important{$\frac{1}{2\pi}\int_{-\pi}^\pi W_\gamma\ejzw\frac{1}{N}|U_N\ejo|^2 d\zeta\approx\sum\limits_{\tau-\gamma}^\gamma \omega_\gamma(\tau)\hat{R}_u(\tau)e^{-j\tau\omega_n}$}

The cross spectral estimate can also be formulated as a convolution in the frequency domain which leads to the analogous formulation to the spectral estimate of $u$:

\important{$\tilde{\phi}_u\ejon=\sum\limits_{\tau=-\gamma}^\gamma \omega_\gamma(\tau)\hat{R}_u(\tau)e^{-j\tau\omega_n}$}

\important{$\tilde{\phi}_{yu}\ejon=\sum\limits_{\tau=-\gamma}^\gamma \omega_\gamma(\tau)\hat{R}_{yu}(\tau)e^{-j\tau\omega_n}$}

\subsubsection{Time domain smoothing in matlab}

\begin{TPMatlab}
gamma = 80;
[~,Wg] = WtHann(gamma,N);
R_u = xcorr(u,N/2)/N; %Lecture 3.37
R_yu = xcorr(y,u,N/2)/N;

omega = Omega_n(N);
phi_u = zeros(size(omega));
phi_yu = zeros(size(omega));

tau = -g:g;
ind = tau+N/2;
for i = 1:N
%Lecture 5.9 and 5.11
    phi_u(i) = sum(Wg(tau+N/2).*R_u(ind).*exp(-1j.*tau.'.*omega(i)));
    phi_yu(i) = sum(Wg(tau+N/2).*R_yu(ind).*exp(-1j*tau.'.*omega(i)));
end
G_est_smooth = phi_yu./phi_u;
\end{TPMatlab}

\subsubsection{Window characteristics}

Decreasing $\gamma$: narrower $\omega_\gamma(\tau)$, wider $W_\gamma\ejo$
\begin{itemize}
\item the more frequencies, $\hat{G}\ejon$ included in the smoothing.
\item the fewer $\hat{R}(\tau)$ estimates included in the smoothing.
\item the smoother the result.
\item the lower the noise induced variance.
\item the \textbf{higher the bias}.
\end{itemize}

\subsection{Input Signals}

\begin{itemize}
\item Steps
\item Doublet
\item Sinusiods, Chirpts, Multi-Sines
\item Filtered white noise
\item Pseudo-Random Binary Signals (PRBS)
\end{itemize}

\subsubsection{PRBS}

\mportant{$u(k)=a\text{ or }-a$}

\mypic{PRBS}

\mportant{$R_u(\tau)=\frac{1}{N}\sum\limits_{k=0}^{N-1}u(k)u(k-\tau)=\begin{cases}a^2&\text{if }\tau=0\\
\frac{-a^2}{2^X-1}&\text{if }\tau\neq 0\end{cases}$}

\begin{define}
\textbf{Run length} defines how long the signal stays high.
\end{define}

The run length distribution of $u(k)$ is then:

\begin{itemize}
\item[] 1/2 runs of length 1
\item[] 1/4 runs of length 2
\item[] 1/8 runs of length 3
\item[] $\vdots$
\end{itemize}

Other properties:

\begin{itemize}
\item Equal energy at all frequencies.
\end{itemize}

\subsubsection{PRBS in matlab}

\begin{TPMatlab}
signal_order = 8;
singal_length = 2.^signal_order-1;
u = idinput(signal_length,'PRBS');
\end{TPMatlab}

\begin{itemize}
\item Note that \verb+idinput+ only allows the choice of the total length of the signal and derives the fitting signal order such that the signal is at least of that chosen length. A warning is made if the chosen signal length is not equivalent to the run length.
\end{itemize}

\subsubsection{Multi-sinusoidal signals}

\mportant{$u(k)=\sum\limits_{s=1}^S\sqrt{2\alpha_s}\cos(\omega_skT+\phi_s)$}

where $T$ is the sampling period, $\omega_s=\frac{2\pi}{T_P}$, $\frac{T_P}{T}=N$, $S\leq\frac{N}{2}$.

\vspace{3ex}

\textbf{Choose $N$ to be a power of 2 for efficient FFT calculations.}

\vspace{3ex}

\mportname{Total signal power}{$\sum\limits_{s=1}^S\alpha_s = 1$}

\textbf{Schroeder phasing}

Select the phases $\phi_s$ such that the minimize the peak amplitude:

\mportant{$\phi_s=2\pi\sum\limits_{j=1}^s j\alpha_s$.}

for equal power in each sinusoids:

\mportant{$\alpha_s=1/S$ and $\phi_s=\frac{\pi(s^2+s)}{S}$}

\section{Residual Spectra, Coherency, Aperiodicty, Offsets and Drifts}

\subsection{Residual Spectrum}

\subsubsection{Estimating $\phi_v\ejon$}

\mportant{$v(k)=y(k)-G\ejo u(k)$}

\important{$\tilde{\phi}_v\ejon\approx\frac{1}{N}\frac{1}{2\pi}\int_{-\pi}^\pi W_\gamma\ejzw\left|Y_N\ejo-\tilde{G}\ejo U_N\ejo\right|^2d\zeta\approx \tilde{\phi}_y\ejon-\frac{\left|\tilde{\phi}_{yu}\ejon\right|^2}{\tilde{\phi}_u\ejon}$}

How much energy is accounted for by the model? How much by noise?

\mportant{$\phi_v\ejon = \phi_y\ejon\left(1-\frac{|\phi_{yu}(\ejon)|^2}{\phi_y\ejon\phi_u\ejon}\right)$}

\importname{Coherency Spectrum}{$\hat{\kappa}_{yu}\ejon=\sqrt{\frac{|\hat{\phi}_{yu}\ejon|^2}{\hat{\phi}_y\ejon\hat{\phi}_u\ejon}}$}

\begin{itemize}
\item If all of the energy in the output is due to the model for a frequency $\omega_n$ then $\hat{\kappa}_{yu}\ejon=1$.
\item This can be used as a measure of effectiveness of the modelling at a particular frequency.
\item Theoretically, $0\leq\hat{\kappa}_{yu}\ejon\leq 1$. One should aim to keep the coherency spectrum as high as possible. It can be adjusted by adjusting the smoothing.
\end{itemize}

\subsection{Time-domain data windowing}

Putting a time domain window directly on the data.

\mportant{$U_w\ejon=\sum\limits_{k=0}^{N-1}w_{data}(k)u(k)e^{-j k\omega_n}$}

often with $w_{data}(k)=w_\gamma(k-N/2)$ (shifted to middle). Typically $\gamma = N/2$ such that all of the data is used.

\myspic{0.5}{TimeDomainWindowing}

For an estimation of the periodogram \textbf{scaling is necessary!}

\importname{Periodogram}{$\frac{1}{E_{scl}}\frac{1}{N} |U_w\ejon|^2\quad\text{where }E_{scl}=\frac{\sum\limits_{k=0}^{N-1}|w_{data}(k)u(k)|^2}{\sum\limits_{k=0}^{N-1}|u(k)|^2}\approx \frac{1}{N}\sum\limits_{k=0}^{N-1}|w_{data}(k)|^2$}

\begin{itemize}
\item Transients influence the periodogram of a sequence, since by the application of the DFT, the signal is assumed to be periodic. Thus even if a sinusoid is sampled - if the periodic extension does not represent the same sinusoid - the DFT will show a range of frequencies, instead of a single distinct one.
\item Time-domain data windowing can help reducing the influence of transients.
\end{itemize}

\subsubsection{Welch's Method}

\myspic{0.5}{Welch}

\begin{enumerate}
\item Split the data record into $L$ overlapping segments of length $N$.
\item $U_l\ejon=\sum\limits_{k=0}^{N-1}w_{data}(k)u_l(k)e^{j\omega_n k}$
\item $\tilde{\phi}_u\ejon=\frac{1}{NLE_{scl}}\sum\limits_{l=1}^L\left|U_l\ejon\right|^2$
\end{enumerate}

\begin{itemize}
\item[+] Windowing can reduce transient response effects.
\item[+] Noise reduction from averaging and windowing.
\item[+] Variance error can be reduced.

\item[-] Windowing can cause energy leakage to adjacent frequencies.
\item[-] Frequency resolution deteriorates.
\item[-] Bias error can be increased.
\item[-] Noise on $u_l(k)$ and $u_{l+1}$ is not uncorrelated.
\item Do not use \verb+welch()+, since it does not fit the definition here.
\end{itemize}

\subsubsection{Drifts and Offsets}

\begin{itemize}
\item Time domain windowing when an offset is present can lead to the introduction of additional frequencies close to zero, which is undesirable.
\item Time domain windowing when a drift is present can lead to the introduction of additional frequencies close to the peak of the sampled sinusoid, which is undesirable as well.
\item A possible solution is \textbf{preprocessing} the data via the assumption

\mportant{$u_d(k)=u(k)-(alpha k+\beta)$}

where $\alpha k+\beta$ is the best linear fit to $u(k)$.
\end{itemize}

\begin{TPMatlab}
detrend(u);
\end{TPMatlab}

\begin{itemize}
\item[+] \verb+detrend+ can completely remove the effect of drifts and offsets.
\item[-] It must be applied carefully, since if no drift/offset is present or if it is detected wrongly, large errors can be introduced.
\end{itemize}

\section{Frequency Domain Subspace ID}

\section{Closed-Loop ID}

\section{Time-Domain Correlation Method}

\section{Prediction Error Methods}

\section{Parameter estimation statistics}

\section{Nomenclature}

\begin{TDefinitionTable*}
$y(k)=Gu(k)$&output signal&$\siu{\ }$\\
$u(k)$&input signal&$\siu{\ }$\\
$G$&plant&$\siu{\ }$\\
$\hat{G}=\frac{y}{u}$&estimated plant&$\siu{\ }$\\
$Y\ejo$&output spectrum&$\siu{\ }$\\
$U\ejo$&input spectrum&$\siu{\ }$\\
ZOH&zero order hold&\\
DAC&digital analog converter&\\
ADC&analog digital converter&\\

\end{TDefinitionTable*}

\end{multicols*}
\end{document}